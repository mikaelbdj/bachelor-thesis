\documentclass[../master/master.tex]{subfiles}
\begin{document}

This section discusses the notions used in the two algorithms we will be focusing on. For the following definitons, let $G=(V,E)$ be a directed graph.

\subsection{Strongly Connected Components}
A strongly connected component (SCC) is a maximal set of nodes $C\in V$ such that $\forall v,w\in C$, there is a path from $v$ to $w$. In other words, a set of nodes in which every node is reachable from every other node.
\begin{figure}[H]
\center
\begin{tikzpicture}
\begin{scope}[every node/.style={circle,thick,draw}]
    \node [fill={red!30}] (A) at (0,0) {0};
    \node [fill={blue!30}](B) at (3,0) {1};
    \node [fill={blue!30}](C) at (6,0) {2};
    \node [fill={green!30}](D) at (9,0) {3};
\end{scope}

\begin{scope}[>={Stealth[black]}]
    \path [->] (A) edge  (B);
    \path [->] (B) edge[bend left=30]  (C);
    \path [->] (C) edge[bend left=30]  (B);
    \path [->] (C) edge  (D);
\end{scope}
\end{tikzpicture}
\caption{A small graph containing three SCCs (marked with different colors)}
\end{figure}

\subsection{Binary decision diagrams}
In this thesis, our primary data structure is a Binary Decision Diagram (BDD). Binary Decision Diagrams represent boolean functions as directed graphs. Each non-terminal vertex $v$ asks about the value of some variable $x_i\in \{0,1\}$ and has edges to two children. The two edges $lo(v)$ and $hi(v)$ correspond to the cases where the variable of $x_i$ is assigned to 0 and 1 respectively. There are two special terminal nodes $0$ and $1$ - these have no children and all maximal paths end in one of these. Note that this is not a tree, since we can have multiple edges going to the same vertex; this exact property is what allows BDDs to be more efficient at storing boolean functions as opposed to storing the boolean functions explicitly.
Given a boolean function $f: \{0,1\}^n \rightarrow \{0,1\}$ we can construct a binary decision diagram representing it. To compute an output $y\in \{0,1\}$ given an assignment  $x\in\{0,1\}^n$, we start in the root and until we hit a terminal vertex we iterate the following: for each vertex $v$ we encounter (including the root) with variable $x_i$, we branch to $lo(v)$ and $hi(v)$ depending on the value assigned to $x_i$. When illustrating BDDs, $hi(v)$ and $lo(v)$ are represented by respectively a full line and a dotted line. A small example of a BDD can be seen in Figure \ref{simplebdd}.

\begin{figure}[H]
\center
\begin{tikzpicture}
\begin{scope}[every node/.style={circle,thick,draw}]
    \node (A) at (0,0) {0};
    \node (B) at (0.5,-1.2) {1};
    \node (C) at (-0.5,-1.2) {1};
    \node[style={rectangle}] (D) at (0.5,-2.4) {0};
    \node[style={rectangle}] (E) at (-0.5,-2.4) {1};
\end{scope}

\begin{scope}[>={Stealth[black]}]
    \path [->,dashed](A) edge  (B);
    \path [->] (A) edge  (C);
    \path [->,dashed](B) edge  (D);
    \path [->] (B) edge  (E);
    \path [->,dashed](C) edge  (E);
    \path [->] (C) edge  (D);
\end{scope}
\end{tikzpicture}
\caption{Small BDD computing the boolean function $\text{XOR}:\{0,1\}^2 \rightarrow \{0,1\}$}
\label{simplebdd}
\end{figure}

BDDs support simple set operations such as union, intersection, difference and equality. Using hashing and heuristics to order variables in an effective way, these operations are computationally cheap.

\subsection{Symbolic Notation}
Symbolic algorithms manipulate sets of nodes by using symbolic operators. For a directed graph G and $S\subseteq V$, we can derive the following operators: 
\begin{align*}
\img S &= \set{v' \in V}{\exists v\in S:(v, v')\in E}{|}\\
\pre S &= \set{v' \in V}{\exists v\in S:(v', v)\in E}{|}
\end{align*}

\begin{figure}[H]
\center
\begin{tikzpicture}
\begin{scope}[every node/.style={circle,thick,draw}]
    \node [fill={red!30}] (A) at (0,0) {0};
    \node [fill={blue!30}](B) at (3,0) {1};
    \node (C) at (6,0) {2};
    \node [fill={blue!30}](D) at (9,0) {3};
\end{scope}

\begin{scope}[>={Stealth[black]}]
    \path [->] (A) edge  (B);
    \path [->] (B) edge[bend left=30]  (C);
    \path [->] (C) edge[bend left=30]  (B);
    \path [->] (C) edge  (D);
\end{scope}
\end{tikzpicture}
\caption{red: $\pre {\{1\}} = \{0\}$, blue: $\img {\{2\}} = \{1,3\}$ }
\label{imgpreimg}
\end{figure}

The image and preimage operations will be refered to as ``steps'', and the complexity of the algorithms will be measured in terms of these steps. From these operators, we can derive the notion of forward and backward sets, which will be key elements in the algorithms.

A forward set, \FW{v}, is the set of nodes that can be reached from $v$. Similarily, a backward set, \BW{v}, is a set if all the nodes that can reach $v$. These can be expressed formally as the least fixpoint of respectively $\lambda S.\{v\}\cup \text{img}(S)$ and $\lambda S.\{v\}\cup \text{preimg}(S)$. Using mu-calculus \cite{clarke_peled_grumberg_1999} this can be expressed as 
\begin{align}\label{fw}
\FW{v} &= \mu S.\{v\}\cup \text{img}_G(S)\\
\label{bw}\BW{v} &= \mu S.\{v\}\cup \text{preimg}_G(S)
\end{align}

\begin{figure}[H]
\center
\begin{tikzpicture}
\begin{scope}[every node/.style={circle,thick,draw}]
    \node (A) at (0,0) {0};
    \node (B) at (3,0) {1};
    \node (C) at (6,0) {2};
    \node (D) at (9,0) {3};
    \node [style={ellipse},fit=(B) (C) (D), fill={blue}, draw=none, opacity=0.3] {};
\end{scope}

\begin{scope}[>={Stealth[black]}]
    \path [->] (A) edge  (B);
    \path [->] (B) edge[bend left=30]  (C);
    \path [->] (C) edge[bend left=30]  (B);
    \path [->] (C) edge  (D);
\end{scope}
\end{tikzpicture}
\begin{tikzpicture}
\begin{scope}[every node/.style={circle,thick,draw}]
    \node (A) at (0,0) {0};
    \node (B) at (3,0) {1};
    \node (C) at (6,0) {2};
    \node (D) at (9,0) {3};
    \node [style={ellipse},fit=(A) (B) (C), fill={red}, draw=none, opacity=0.3] {};
\end{scope}

\begin{scope}[>={Stealth[black]}]
    \path [->] (A) edge  (B);
    \path [->] (B) edge[bend left=30]  (C);
    \path [->] (C) edge[bend left=30]  (B);
    \path [->] (C) edge  (D);
\end{scope}
\end{tikzpicture}
\caption{The forward set (blue) and backward set (red) of the node $1$ }
\label{fwbw}
\end{figure}


\noindent Examining the definitions of forward and backward sets, the following is trivial:
\begin{align}
SCC_G(v) = \FW{v} \cap \BW{v} \label{scc}
\end{align} 
where $SCC_G(v)$ is the SCC in $G$ containing the node $v$. This can be illustrated nicely by overlaying the forward set and backward set that we showed in Figure \ref{fwbw}.


\begin{figure}[H]
\center
\begin{tikzpicture}
\begin{scope}[every node/.style={circle,thick,draw}]
    \node (A) at (0,0) {0};
    \node (B) at (3,0) {1};
    \node (C) at (6,0) {2};
    \node (D) at (9,0) {3};
    \node [style={ellipse},fit=(A) (B) (C), fill={red}, draw=none, opacity=0.3] {};
    \node [style={ellipse},fit=(B) (C) (D), fill={blue}, draw=none, opacity=0.3] {};
\end{scope}

\begin{scope}[>={Stealth[black]}]
    \path [->] (A) edge  (B);
    \path [->] (B) edge[bend left=30]  (C);
    \path [->] (C) edge[bend left=30]  (B);
    \path [->] (C) edge  (D);
\end{scope}
\end{tikzpicture}
\caption{Intersection of $\FW{v}$ and $\BW{v}$ gives us SCC containing node 1; indeed $SCC_G(1) = \{1,2\}$. } 
\end{figure}

The authors in \cite{linear} introduce a notion of a skeleton and a spine-set. First, we need to define a chordless path. A chordless path is a path between nodes $v_0$ and $v_p$ iff for all $0\leq i \leq j \leq p$, where $j-i>1$ and there is no edge from $v_i$ and $v_j$\todo{maybe explain more intuitively if possible}.

A spine-set is a symbolic way of ordering sets of nodes. A pair \pair{S, v} is a spine-set of G iff G contains a chordless path consisting of the set of vertices $S$, ending in $v$. $v$ can also be called the spine-anchor of the spine-set.

\begin{figure}[H]
\center
\begin{tikzpicture}
  roundnode/.style={circle, fill=black, inner sep=0pt, minimum size=2mm}
\begin{scope}[every node/.style={circle, fill=black, inner sep=0pt, minimum size=2mm}]
    \node (A) at (0,0) {00};
    \node (B) at (3,0) {01};
    \node (C) at (6,0) {10};
    \node (D) at (9,0) {11};
\end{scope}

\begin{scope}[>={Stealth[black]}]
    \path [->] (A) edge  (B);
    \path [->] (B) edge[bend left=30]  (C);
    \path [->] (C) edge[bend left=30]  (B);
    \path [->] (C) edge  (D);
\end{scope}
\end{tikzpicture}
\caption{Spine-set}
\label{spine}
\end{figure}

For $u, v \in V$ and a forward set \FW{v}, the pair \pair{S}{v} is a skeleton iff $u$ is also a node in $\FW{u}$ and the distance from $v$ to $u$ is maximum and $S$ is the set of nodes on the shortest path between the two vertices.

If \FW{v} is the forward-set of $v \in V$, and if the skeleton of this forward set is \pair{S}{u}, then it is also a spine-set in G. We will be making use of this in the linear-time algorithm when computing the spine-set. \todo{add illustration Gentilini p. 139-141}

\subsection{Representing a graph symbolically}
A BDD can be used to represent many things - in our paper we use it to represent a graph. The common way to represent a graph $G = (V, E)$ by a BDD is to assign each $v\in V$ to be a unique vector of boolean assignments $x\in\{0,1\}^n$. To name all nodes we would trivially need an $n \geq \ceil{\log_2{|V|}}$. Taking our graph from before we can name each node using two booleans:

\begin{figure}[H]
\center
\begin{tikzpicture}
\begin{scope}[every node/.style={circle,thick,draw}]
    \node (A) at (0,0) {00};
    \node (B) at (3,0) {01};
    \node (C) at (6,0) {10};
    \node (D) at (9,0) {11};
\end{scope}

\begin{scope}[>={Stealth[black]}]
    \path [->] (A) edge  (B);
    \path [->] (B) edge[bend left=30]  (C);
    \path [->] (C) edge[bend left=30]  (B);
    \path [->] (C) edge  (D);
\end{scope}
\end{tikzpicture}
\caption{The graph from before, now with binary node naming}
\label{graphforbdd}
\end{figure}

A BDD representing all the nodes is simply a BDD that satisfies each boolean assignment $x\in\{0,1\}^n$ that corresponds to a $v\in V$. Similarly, an edge can be represented by a boolean assignment $x\in\{0,1\}^{2n}$, which is the boolean assignment of two nodes concatenated. Thus, a BDD representing all edges is one that satisfies every $x\in\{0,1\}^{2n}$ that corresponds to a $(v_i, v_j)\in E$.

As an example here are two small BDDs representing the edges $(00,01)$ and $(01,10)$:

\begin{figure}[H]
\center
\begin{tikzpicture}
\begin{scope}[every node/.style={circle,thick,draw}]
    \node (A) at (0,0) {0};
    \node (B) at (0,-1.2) {1};
    \node (C) at (0,-2.4) {2};
    \node (D) at (0,-3.6) {3};
    \node[style={rectangle}] (E) at (2,-2.4) {0};
    \node[style={rectangle}] (F) at (2,-3.6) {1};
\end{scope}

\begin{scope}[>={Stealth[black]}]
    \path [->,dashed] (A) edge  (B);
    \path [->,dashed] (B) edge  (C);
    \path [->,dashed] (C) edge  (D);
    \path [->] (A) edge  (E);
    \path [->] (B) edge  (E);
    \path [->] (C) edge  (E);
    \path [->] (D) edge  (F);
    \path [->,dashed] (D) edge  (E);
\end{scope}
\end{tikzpicture}
\hspace*{5mm}
\begin{tikzpicture}
\begin{scope}[every node/.style={circle,thick,draw}]
    \node (A) at (0,0) {0};
    \node (B) at (0,-1.2) {1};
    \node (C) at (0,-2.4) {2};
    \node (D) at (0,-3.6) {3};
    \node[style={rectangle}] (E) at (2,-2.4) {0};
    \node[style={rectangle}] (F) at (2,-3.6) {1};
\end{scope}

\begin{scope}[>={Stealth[black]}]
    \path [->,dashed] (A) edge  (B);
    \path [->] (B) edge  (C);
    \path [->] (C) edge  (D);
    \path [->] (A) edge  (E);
    \path [->,dashed] (B) edge  (E);
    \path [->,dashed] (C) edge  (E);
    \path [->,dashed] (D) edge  (F);
    \path [->] (D) edge  (E);
\end{scope}
\end{tikzpicture}
\caption{Symbolic representations (via BDDs) of the sets $\{0001\}$ (left) and $\{0110\}$ (right)}
\end{figure}

These should be seen as symbolical representations of sets (sets consisting of all satisfying assignments). To compute the union of two sets symbolically, we can mechanically compute the OR of two BDDs as illustrated:

\begin{figure}[H]
\center
\begin{tikzpicture}
\begin{scope}[every node/.style={circle,thick,draw}]
    \node (A) at (0,0) {0};
    \node (B) at (-1.5,-1.2) {1};
    \node (C) at (0,-1.2) {2};
    \node (D) at (-1.5,-2.4) {2};
    \node (E) at (0,-2.4) {3};
    \node (F) at (-1.5,-3.6) {3}; 
    \node[style={rectangle}] (G) at (0,-4.8) {0};
    \node[style={rectangle}] (H) at (-1.5,-4.8) {1};
\end{scope}

\begin{scope}[>={Stealth[black]}]
    \path [->,dashed] (A) edge  (B);
    \path [->,dashed] (B) edge  (C);
    \path [->] (B) edge  (D);
    \path [->,dashed] (C) edge  (E);
    \path [->] (D) edge  (F);
    \path [->] (A) edge[bend left=50]   (G);
    \path [->] (C) edge[bend left=30]  (G);
    \path [->,dashed] (E) edge  (G);
    \path [->,dashed] (D) edge  (G);
    \path [->,dashed] (F) edge  (H);
    \path [->] (E) edge  (H);
\end{scope}
\end{tikzpicture}
\caption{Symbolic representation of the set $\{0001,0110\}$.}
\end{figure}

Using this process we can construct a BDD representing the set of all edges in our simple graph from Figure \ref{graphforbdd}, whcih can be seen in Figure \ref{edgebdd}.

\begin{figure}[H]
\center
\begin{tikzpicture}
\begin{scope}[every node/.style={circle,thick,draw}]
    \node (A) at (0,0) {0};
    \node (B) at (-0.5,-1.2) {1};
    \node (C) at (0.5,-1.2) {1};
    \node (D) at (0.5,-2.4) {2};
    \node (E) at (1.5,-2.4) {2};
    \node (F) at (-0.5,-4) {3};
    \node (G) at (0.5,-4) {3};
    \node[style={rectangle}] (H) at (-0.5,-5.5) {0};
    \node[style={rectangle}] (I) at (0.5,-5.5) {1};
\end{scope}

\begin{scope}[>={Stealth[black]}]
    \path [->] (A) edge  (B);
    \path [->,dashed] (A) edge  (C);
    \path [->] (C) edge  (D);
    \path [->,dashed] (C) edge  (E);
    \path [->,dashed] (B) edge  (F);
    \path [->] (D) edge  (G);
    \path [->,dashed] (E) edge  (F);
    \path [->,dashed] (F) edge  (H);
    \path [->] (F) edge  (I);
    \path [->] (G) edge  (H);
    \path [->,dashed] (G) edge  (I);
    \path [->,dashed] (D) edge  (H);
    \path [->] (E) edge[bend left=20]  (H);
    \path [->] (B) edge[bend right=40]  (H);
\end{scope}
\end{tikzpicture}
\caption{Symbolic representation of the set $\{0001,0110,1001,1011\}$. This corresponds to all edges of our graph in Figure \ref{graphforbdd}. Since all nodes are part of an edge, this BDD can fully symbolically represent the original graph.}
\label{edgebdd}
\end{figure}

This representation makes it easy to compute useful properties. For example, assume we have a set of nodes $S\subseteq V$ and a set of edges $E$ represented by BDDs $B_S$ and $B_E$. We can compute the image of $S$ by restricting the first $n$ variables of $B_E$ to all satisfying assignments $x\in\{0,1\}^{n}$ to $B_S$. Doing this effectively selects the edges in $E$ that start in $S$. To return the nodes we remove the first $n$ variables, and then rename the last $n$ variables to their counterparts in the first $n$ variables (subtract $n$ from every variable). We are left with a BDD representing a set of nodes $S_{img}\subseteq V$, which is exactly the image of $S$. In terms of BDD operations, this process is done by computing the intersection $B_{E}' = B_S\cap B_E$, then removing the first $n$ variables of $B_E'$. Computation of pre-image is similar with the addition that we have to rename the variables of $B_S$ to correspond to the \textbf{last} $n$ variables of $B_E$.

As an example, let us compute the image of node $10$, as we did in the original graph in Figure \ref{imgpreimg} (blue). The BDD representing $\{10\}$ is simple:

\begin{figure}[H]
\center
\begin{tikzpicture}
\begin{scope}[every node/.style={circle,thick,draw}]
    \node (A) at (0,0) {0};
    \node (B) at (0,-1.2) {1};
    \node[style={rectangle}] (C) at (-0.5,-2.4) {1};
    \node[style={rectangle}] (D) at (0.5,-2.4) {0};
\end{scope}

\begin{scope}[>={Stealth[black]}]
    \path [->](A) edge  (B);
    \path [->,dashed](B) edge  (C);
    \path [->](B) edge  (D);
    \path [->,dashed](A) edge[bend left=30]  (D);
\end{scope}
\end{tikzpicture}
\caption{Symbollic representation of $\{10\}$}
\end{figure}
Now compute the intersection of this and the BDD we constructed of all edges in Figure \ref{edgebdd}. Remove the first two variables and rename the variables accordingly as described before. These three steps are illustrated in Figure  \ref{imgprocess}:

\begin{figure}[H]
\begin{subfigure}[b]{0.3\textwidth}
\centering
\begin{tikzpicture}
\begin{scope}[every node/.style={circle,thick,draw}]
    \node (A) at (0,0) {0};
    \node (B) at (0,-1.2) {1};
    \node (C) at (0,-2.4) {3};
    \node[style={rectangle}] (D) at (-0.5,-3.6) {1};
    \node[style={rectangle}] (E) at (0.5,-3.6) {0};
\end{scope}

\begin{scope}[>={Stealth[black]}]
    \path [->](A) edge  (B);
    \path [->,dashed](B) edge  (C);
    \path [->](B) edge[bend left=20]  (E);
    \path [->,dashed](A) edge[bend left=30]  (E);
    \path [->,dashed](C) edge  (E);
    \path [->](C) edge  (D);
\end{scope}
\end{tikzpicture}
         \caption{$B_E' = B_E\cap B_S$ - Done by ANDing the two BDDs.}
\end{subfigure}
\hfill
\begin{subfigure}[b]{0.3\textwidth}
\centering
\begin{tikzpicture}
\begin{scope}[every node/.style={circle,thick,draw}]
    \node (C) at (0,-2.4) {3};
    \node[style={rectangle}] (D) at (-0.5,-3.6) {1};
    \node[style={rectangle}] (E) at (0.5,-3.6) {0};
\end{scope}

\begin{scope}[>={Stealth[black]}]
    \path [->,dashed](C) edge  (E);
    \path [->](C) edge  (D);
\end{scope}
\end{tikzpicture}
         \caption{$B_E''$. Remove the first 2 variables}
\end{subfigure}
\hfill
\begin{subfigure}[b]{0.3\textwidth}
\centering
\begin{tikzpicture}
\begin{scope}[every node/.style={circle,thick,draw}]
    \node (C) at (0,-2.4) {1};
    \node[style={rectangle}] (D) at (-0.5,-3.6) {1};
    \node[style={rectangle}] (E) at (0.5,-3.6) {0};
\end{scope}

\begin{scope}[>={Stealth[black]}]
    \path [->,dashed](C) edge  (E);
    \path [->](C) edge  (D);
\end{scope}
\end{tikzpicture}
         \caption{$B_{img} = B_E''[2\rightarrow 0, 3\rightarrow 1]$}. Rename the variables
         \label{resultingBdd}
\end{subfigure}
\center

\caption{The full process of computing the image of $S=\{10\}$. }
\label{imgprocess}
\end{figure}

Indeed if we look at the resulting BDD in Figure \ref{resultingBdd}, it symbolically represents the set $\{01,11\}$, which is exactly corresponds to our explicit findings earlier in Figure \ref{imgpreimg} ($\{1,3\}$).


\end{document}
%%% Local Variables:
%%% mode: latex
%%% TeX-master: "../master/master"
%%% End:
