\documentclass[../master/master.tex]{subfiles}
\begin{document}

This section discusses the notions used in the two algorithms we will be focusing on. In the following definitons, let $G=(V,E)$ be a directed graph.

\subsection{Strongly Connected Components}
A strongly connected component (SCC) is a maximal set of nodes $C\subseteq V$, such that $\forall v,w\in C$, there is a path from $v$ to $w$. In other words, a set of nodes ..." => "In other words, every node in an SCC is reachable from every other node in the same SCC. A trivial SCC is an SCC that only contains a single node.
\begin{figure}[H]
\center
\begin{tikzpicture}
\begin{scope}[every node/.style={circle,thick,draw}]
    \node [fill={red!30}] (A) at (0,0) {0};
    \node [fill={blue!30}](B) at (3,0) {1};
    \node [fill={blue!30}](C) at (6,0) {2};
    \node [fill={green!30}](D) at (9,0) {3};
\end{scope}

\begin{scope}[>={Stealth[black]}]
    \path [->] (A) edge  (B);
    \path [->] (B) edge[bend left=30]  (C);
    \path [->] (C) edge[bend left=30]  (B);
    \path [->] (C) edge  (D);
\end{scope}
\end{tikzpicture}
\caption{A small graph containing three SCCs (marked with different colors) - (0) and (3) are trivial}
\end{figure}

\subsection{Binary Decision Diagrams}
\todo{ Den første paragraf kunne forbedres ved at referere til de to reduktionsregler af Bryant fra 1986. Begge regler kan ses i lyset af at komprimere grafen og spare plads.}
A Binary Decision Diagram represents a boolean function as a directed acyclic graph (DAGs). Each non-terminal vertex $v$ represents a choice based on the value of some input variable $x_i\in \{0,1\}$ and has edges to two children. The two edges $lo(v)$ and $hi(v)$ correspond to the cases where the variable of $x_i$ is assigned to 0 and 1 respectively. There are two special terminal nodes $0$ and $1$ - these have no children and all maximal paths end in one of these. One node, potentially a terminal node, has no ingoing edges and is the root of the DAG. Note that the graph is not a tree, since we can have multiple edges going to the same vertex; this exact property is what allows BDDs to be more efficient at storing boolean functions as opposed to storing the boolean functions explicitly.
\todo[inline]{ "Given a boolean function f, we can construct a BDD representing it. To compute an output y given an assignment x, ..."

  Den første sætning lægger op til funktion -> BDD, men i beskriver i den næste sætning hvordan man kan interpretere en BDD som en funktion (altså baglæns). Der mangler enten et "as described by Bryant [1986]" eller direkte at gentage argumentationen ved hjælp af reduktionsreglerne.

  "A binary tree of depth $n$ represents $f : \{0,1\}^n \rightarrow \{0,1\}$ by  having each path describe the assignment to the input variables $x_1, x_2, \dots, x_n$ ; the two reduction rules of Bryant}
Given a boolean function $f: \{0,1\}^n \rightarrow \{0,1\}$ we can construct a binary decision diagram that represents it. To compute an output $y\in \{0,1\}$ given an assignment  $x\in\{0,1\}^n$, we start in the root and until we hit a terminal vertex, we iterate the following: for each vertex $v$ we encounter (including the root) with variable $x_i$, we branch to $lo(v)$ or $hi(v)$ depending on the value assigned to $x_i$. When illustrating BDDs, $hi(v)$ and $lo(v)$ are represented by respectively a full line and a dotted line. A small example of a BDD can be seen in Figure \ref{simplebdd}.

\begin{figure}[H]
\center
\begin{tikzpicture}
\begin{scope}[every node/.style={circle,thick,draw}]
    \node (A) at (0,0) {0};
    \node (B) at (0.5,-1.2) {1};
    \node (C) at (-0.5,-1.2) {1};
    \node[style={rectangle}] (D) at (0.5,-2.4) {0};
    \node[style={rectangle}] (E) at (-0.5,-2.4) {1};
\end{scope}

\begin{scope}[>={Stealth[black]}]
    \path [->,dashed](A) edge  (B);
    \path [->] (A) edge  (C);
    \path [->,dashed](B) edge  (D);
    \path [->] (B) edge  (E);
    \path [->,dashed](C) edge  (E);
    \path [->] (C) edge  (D);
\end{scope}
\end{tikzpicture}
\caption{Small BDD computing the boolean function $\text{XOR}:\{0,1\}^2 \rightarrow \{0,1\}$}
\label{simplebdd}
\end{figure}

As BDDs represent boolean functions, they can also represent a set $S$ via the indicator function on a binary encoding of each element of the universe. Thus, a BDD $B_S$ representing the indicator function $f_S$ can also represent the set $S$. In other words, we can think of a BDD with cardinality $n$ as representing the set of all assignments $x_n$ that satisfies it. Going forward with this, we can define set operations on BDDs $B_A$ and $B_B$:
\begin{itemize}
\item $B_A\cup B_B$: This corresponds to computing the logical OR of $B_A$ and $B_B$.
\item $B_A\cap B_B$: This corresponds to computing the logical AND of $B_A$ and $B_B$.
\item $B_A\setminus B_B$: This corresponds to computing the logical AND of $B_A$ and NOT $B_B$.
\end{itemize}

Computing these logical operations is a mechanical process that is computationally cheap. For the BDDs to preserve their efficient nature, some variable ordering heuristics should be done after each operation. We will make use of multiple well established optimisations for the implementation of BDDs, such as unique nodes managed in a hash table \cite{hashing1, hashing2} for fast equality comparisons, dynamic variable reordering \cite{buddy} and \textit{reductions} \cite{bryant_1992}. 

\subsection{Symbolic Notation}
Symbolic algorithms manipulate sets of nodes by using symbolic operators. For a directed graph G and $S\subseteq V$, we can derive the following operators: 
\begin{align*}
\img S &= \set{v' \in V}{\exists v\in S:(v, v')\in E}{|}\\
\pre S &= \set{v' \in V}{\exists v\in S:(v', v)\in E}{|}
\end{align*}

\begin{figure}[H]
\center
\begin{tikzpicture}
\begin{scope}[every node/.style={circle,thick,draw}]
    \node [fill={red!30}] (A) at (0,0) {0};
    \node [fill={blue!30}](B) at (3,0) {1};
    \node (C) at (6,0) {2};
    \node [fill={blue!30}](D) at (9,0) {3};
\end{scope}

\begin{scope}[>={Stealth[black]}]
    \path [->] (A) edge  (B);
    \path [->] (B) edge[bend left=30]  (C);
    \path [->] (C) edge[bend left=30]  (B);
    \path [->] (C) edge  (D);
\end{scope}
\end{tikzpicture}
\caption{red: $\pre {\{1\}} = \{0\}$, blue: $\img {\{2\}} = \{1,3\}$ }
\label{imgpreimg}
\end{figure}

The image and preimage operations will be refered to as ``steps'', and the complexity of the algorithms will be measured by the number of steps taken. From these operators, we can derive the notion of forward and backward sets, which will be key elements in the algorithms.

A forward set, \FW{v}, is the set of nodes that can be reached from $v$. Similarily, a backward set, \BW{v}, is a set of all the nodes that can reach $v$. These can be expressed formally as the least fixpoint of respectively $\lambda S.\{v\}\cup \text{img}(S)$ and $\lambda S.\{v\}\cup \text{preimg}(S)$. Using mu-calculus \cite{clarke_peled_grumberg_1999} this can be expressed as 
\begin{align}\label{fw}
\FW{v} &= \mu S.\{v\}\cup \text{img}_G(S)\\
\label{bw}\BW{v} &= \mu S.\{v\}\cup \text{preimg}_G(S)
\end{align}

\begin{figure}[H]
\center
\begin{tikzpicture}
\begin{scope}[every node/.style={circle,thick,draw}]
    \node (A) at (0,0) {0};
    \node (B) at (3,0) {1};
    \node (C) at (6,0) {2};
    \node (D) at (9,0) {3};
    \node [style={ellipse},fit=(B) (C) (D), fill={blue}, draw=none, opacity=0.3] {};
\end{scope}

\begin{scope}[>={Stealth[black]}]
    \path [->] (A) edge  (B);
    \path [->] (B) edge[bend left=30]  (C);
    \path [->] (C) edge[bend left=30]  (B);
    \path [->] (C) edge  (D);
\end{scope}
\end{tikzpicture}
\begin{tikzpicture}
\begin{scope}[every node/.style={circle,thick,draw}]
    \node (A) at (0,0) {0};
    \node (B) at (3,0) {1};
    \node (C) at (6,0) {2};
    \node (D) at (9,0) {3};
    \node [style={ellipse},fit=(A) (B) (C), fill={red}, draw=none, opacity=0.3] {};
\end{scope}

\begin{scope}[>={Stealth[black]}]
    \path [->] (A) edge  (B);
    \path [->] (B) edge[bend left=30]  (C);
    \path [->] (C) edge[bend left=30]  (B);
    \path [->] (C) edge  (D);
\end{scope}
\end{tikzpicture}
\caption{The forward set (blue) and backward set (red) of the node $1$ }
\label{fwbw}
\end{figure}


\noindent Examining the definitions of forward and backward sets, the following is trivial:
\begin{align}
SCC_G(v) = \FW{v} \cap \BW{v} \label{scc}
\end{align} 
where $SCC_G(v)$ is the SCC in $G$ containing the node $v$. This is illustrated in Figure \ref{overlayedfwbw}, where the intersection is overlayed on top of the forward and backward set in Figure \ref{fwbw}.


\begin{figure}[H]
\center
\begin{tikzpicture}
\begin{scope}[every node/.style={circle,thick,draw}]
    \node (A) at (0,0) {0};
    \node (B) at (3,0) {1};
    \node (C) at (6,0) {2};
    \node (D) at (9,0) {3};
    \node [style={ellipse},fit=(A) (B) (C), fill={red}, draw=none, opacity=0.3] {};
    \node [style={ellipse},fit=(B) (C) (D), fill={blue}, draw=none, opacity=0.3] {};
\end{scope}

\begin{scope}[>={Stealth[black]}]
    \path [->] (A) edge  (B);
    \path [->] (B) edge[bend left=30]  (C);
    \path [->] (C) edge[bend left=30]  (B);
    \path [->] (C) edge  (D);
\end{scope}
\end{tikzpicture}
\caption{Intersection of $\FW{v}$ and $\BW{v}$ gives us SCC containing node 1; indeed $SCC_G(1) = \{1,2\}$. } \label{overlayedfwbw}
\end{figure}


\subsection{Representing a graph symbolically}
A BDD can be used to represent many things - for our purposes, we use it to represent a graph. The common way to represent a graph $G$ as a BDD is to assign each $v\in V$ to be a unique vector of boolean assignments $x\in\{0,1\}^n$. To name all nodes, we need an $n \geq \ceil{\log_2{|V|}}$ number of variables. Revisiting our graph from Figure \ref{imgpreimg}, we can name each node using two booleans:

\begin{figure}[H]
\center
\begin{tikzpicture}
\begin{scope}[every node/.style={circle,thick,draw}]
    \node (A) at (0,0) {00};
    \node (B) at (3,0) {01};
    \node (C) at (6,0) {10};
    \node (D) at (9,0) {11};
\end{scope}

\begin{scope}[>={Stealth[black]}]
    \path [->] (A) edge  (B);
    \path [->] (B) edge[bend left=30]  (C);
    \path [->] (C) edge[bend left=30]  (B);
    \path [->] (C) edge  (D);
\end{scope}
\end{tikzpicture}
\caption{The graph from before, now with binary node naming}
\label{graphforbdd}
\end{figure}

A BDD representing all the nodes is simply a BDD that satisfies each boolean assignment $x\in\{0,1\}^n$ that corresponds to a $v\in V$. An edge $(s,t) \in E$ can be represented by a boolean assignment $x\in\{0,1\}^{2n}$, where the first n variables are the bitvector of the source s and the rest identify the target t. This can be viewed as a tuple of two nodes, and  using this term we can view the BDD as representing a relation over $V\times V$. 
Thus, a BDD $B_E$ representing edges $E$, is a relation $R_E$ where: $\forall a, b \in V, a\ R_E\ b$ iff. $(a,b)\in E$.

As an example, Figure \ref{bddedges} shows two BDDs representing the edges $(00,01)$ and $(01,10)$:
\todo[inline]{ Figur 7 kan godt bruge to subfigure}
\begin{figure}[H]
\center
\begin{tikzpicture}
\begin{scope}[every node/.style={circle,thick,draw}]
    \node (A) at (0,0) {0};
    \node (B) at (0,-1.2) {1};
    \node (C) at (0,-2.4) {2};
    \node (D) at (0,-3.6) {3};
    \node[style={rectangle}] (E) at (2,-2.4) {0};
    \node[style={rectangle}] (F) at (2,-3.6) {1};
\end{scope}

\begin{scope}[>={Stealth[black]}]
    \path [->,dashed] (A) edge  (B);
    \path [->,dashed] (B) edge  (C);
    \path [->,dashed] (C) edge  (D);
    \path [->] (A) edge  (E);
    \path [->] (B) edge  (E);
    \path [->] (C) edge  (E);
    \path [->] (D) edge  (F);
    \path [->,dashed] (D) edge  (E);
\end{scope}
\end{tikzpicture}
\hspace*{5mm}
\begin{tikzpicture}
\begin{scope}[every node/.style={circle,thick,draw}]
    \node (A) at (0,0) {0};
    \node (B) at (0,-1.2) {1};
    \node (C) at (0,-2.4) {2};
    \node (D) at (0,-3.6) {3};
    \node[style={rectangle}] (E) at (2,-2.4) {0};
    \node[style={rectangle}] (F) at (2,-3.6) {1};
\end{scope}

\begin{scope}[>={Stealth[black]}]
    \path [->,dashed] (A) edge  (B);
    \path [->] (B) edge  (C);
    \path [->] (C) edge  (D);
    \path [->] (A) edge  (E);
    \path [->,dashed] (B) edge  (E);
    \path [->,dashed] (C) edge  (E);
    \path [->,dashed] (D) edge  (F);
    \path [->] (D) edge  (E);
\end{scope}
\end{tikzpicture}
\caption{Symbolic representations (via BDDs) of the relations $\{(00,01)\}$ (left) and $\{(01,10)\}$ (right)} \label{bddedges}
\end{figure}
As mentioned above, to compute the union of two sets/relations symbolically, we can mechanically compute the OR of two BDDs as illustrated:

\begin{figure}[H]
\center
\begin{tikzpicture}
\begin{scope}[every node/.style={circle,thick,draw}]
    \node (A) at (0,0) {0};
    \node (B) at (-1.5,-1.2) {1};
    \node (C) at (0,-1.2) {2};
    \node (D) at (-1.5,-2.4) {2};
    \node (E) at (0,-2.4) {3};
    \node (F) at (-1.5,-3.6) {3}; 
    \node[style={rectangle}] (G) at (0,-4.8) {0};
    \node[style={rectangle}] (H) at (-1.5,-4.8) {1};
\end{scope}

\begin{scope}[>={Stealth[black]}]
    \path [->,dashed] (A) edge  (B);
    \path [->,dashed] (B) edge  (C);
    \path [->] (B) edge  (D);
    \path [->,dashed] (C) edge  (E);
    \path [->] (D) edge  (F);
    \path [->] (A) edge[bend left=50]   (G);
    \path [->] (C) edge[bend left=30]  (G);
    \path [->,dashed] (E) edge  (G);
    \path [->,dashed] (D) edge  (G);
    \path [->,dashed] (F) edge  (H);
    \path [->] (E) edge  (H);
\end{scope}
\end{tikzpicture}
\caption{Symbolic representation of the relation $\{(00,01),(01,10)\}$.}
\end{figure}

Using this process we can construct a BDD representing the set of all edges in our simple graph from Figure \ref{graphforbdd}, which can be seen in Figure \ref{edgebdd}. Since all nodes are part of an edge, this BDD can fully symbolically represent the original graph.

\begin{figure}[H]
\center
\begin{tikzpicture}
\begin{scope}[every node/.style={circle,thick,draw}]
    \node (A) at (0,0) {0};
    \node (B) at (-0.5,-1.2) {1};
    \node (C) at (0.5,-1.2) {1};
    \node (D) at (0.5,-2.4) {2};
    \node (E) at (1.5,-2.4) {2};
    \node (F) at (-0.5,-4) {3};
    \node (G) at (0.5,-4) {3};
    \node[style={rectangle}] (H) at (-0.5,-5.5) {0};
    \node[style={rectangle}] (I) at (0.5,-5.5) {1};
\end{scope}

\begin{scope}[>={Stealth[black]}]
    \path [->] (A) edge  (B);
    \path [->,dashed] (A) edge  (C);
    \path [->] (C) edge  (D);
    \path [->,dashed] (C) edge  (E);
    \path [->,dashed] (B) edge  (F);
    \path [->] (D) edge  (G);
    \path [->,dashed] (E) edge  (F);
    \path [->,dashed] (F) edge  (H);
    \path [->] (F) edge  (I);
    \path [->] (G) edge  (H);
    \path [->,dashed] (G) edge  (I);
    \path [->,dashed] (D) edge  (H);
    \path [->] (E) edge[bend left=20]  (H);
    \path [->] (B) edge[bend right=40]  (H);
\end{scope}
\end{tikzpicture}
\caption{Symbolic representation of the set $\{(00,01),(01,10),(10,01),(10,11)\}$. This corresponds to all edges of our graph in Figure \ref{graphforbdd}.}
\label{edgebdd}
\end{figure}

This representation makes it easy to compute useful properties. For example, assume we have a set of nodes $S\subseteq V$ and a set of edges $E$ represented by BDDs $B_S$ and $B_E$. We can compute the image of $S$ by restricting the first $n$ variables of $B_E$ to all satisfying assignments $x\in\{0,1\}^{n}$ to $B_S$. Doing this effectively selects the edges in $E$ that start in $S$. To return the nodes we remove the first $n$ variables, and then rename the last $n$ variables to their counterparts in the first $n$ variables (subtract $n$ from every variable). We are left with a BDD representing a set of nodes $S_{img}\subseteq V$, which is the image of $S$. In terms of BDD operations, this process is done by computing the intersection $B_{E}' = B_S\cap B_E$, then removing the first $n$ variables of $B_E'$ \todo{ Skal i ikke også her skrive noget om renaming af de sidste variable? At fjerne disse variable virker heller ikke helt trivielt for mig. Hvordan kan i være garanteret en enkelt rod i den nye DAG?}. Computation of pre-image is similar with the addition that we have to rename the variables of $B_S$ to correspond to the \textbf{last} $n$ variables of $B_E$.

As an example, let us compute the image of node $10$, as we did in the original graph in Figure \ref{imgpreimg} (blue). The BDD representing $\{10\}$ is simple:

\begin{figure}[H]
\center
\begin{tikzpicture}
\begin{scope}[every node/.style={circle,thick,draw}]
    \node (A) at (0,0) {0};
    \node (B) at (0,-1.2) {1};
    \node[style={rectangle}] (C) at (-0.5,-2.4) {1};
    \node[style={rectangle}] (D) at (0.5,-2.4) {0};
\end{scope}

\begin{scope}[>={Stealth[black]}]
    \path [->](A) edge  (B);
    \path [->,dashed](B) edge  (C);
    \path [->](B) edge  (D);
    \path [->,dashed](A) edge[bend left=30]  (D);
\end{scope}
\end{tikzpicture}
\caption{Symbolic representation of $\{10\}$}
\end{figure}
Now, compute the intersection of this and the BDD we constructed of all edges in Figure \ref{edgebdd}. Remove the first two variables and rename the variables accordingly, as described before. These three steps are illustrated in Figure  \ref{imgprocess}:

\begin{figure}[H]
\begin{subfigure}[b]{0.3\textwidth}
\centering
\begin{tikzpicture}
\begin{scope}[every node/.style={circle,thick,draw}]
    \node (A) at (0,0) {0};
    \node (B) at (0,-1.2) {1};
    \node (C) at (0,-2.4) {3};
    \node[style={rectangle}] (D) at (-0.5,-3.6) {1};
    \node[style={rectangle}] (E) at (0.5,-3.6) {0};
\end{scope}

\begin{scope}[>={Stealth[black]}]
    \path [->](A) edge  (B);
    \path [->,dashed](B) edge  (C);
    \path [->](B) edge[bend left=20]  (E);
    \path [->,dashed](A) edge[bend left=30]  (E);
    \path [->,dashed](C) edge  (E);
    \path [->](C) edge  (D);
\end{scope}
\end{tikzpicture}
         \caption{$B_E' = B_E\cap B_S$ - Done by ANDing the two BDDs.}
\end{subfigure}
\hfill
\begin{subfigure}[b]{0.3\textwidth}
\centering
\begin{tikzpicture}
\begin{scope}[every node/.style={circle,thick,draw}]
    \node (C) at (0,-2.4) {3};
    \node[style={rectangle}] (D) at (-0.5,-3.6) {1};
    \node[style={rectangle}] (E) at (0.5,-3.6) {0};
\end{scope}

\begin{scope}[>={Stealth[black]}]
    \path [->,dashed](C) edge  (E);
    \path [->](C) edge  (D);
\end{scope}
\end{tikzpicture}
         \caption{$B_E''$. Remove the first 2 variables}
\end{subfigure}
\hfill
\begin{subfigure}[b]{0.3\textwidth}
\centering
\begin{tikzpicture}
\begin{scope}[every node/.style={circle,thick,draw}]
    \node (C) at (0,-2.4) {1};
    \node[style={rectangle}] (D) at (-0.5,-3.6) {1};
    \node[style={rectangle}] (E) at (0.5,-3.6) {0};
\end{scope}

\begin{scope}[>={Stealth[black]}]
    \path [->,dashed](C) edge  (E);
    \path [->](C) edge  (D);
\end{scope}
\end{tikzpicture}
         \caption{$B_{img} = B_E''[2\rightarrow 0, 3\rightarrow 1]$}. Rename the variables
         \label{resultingBdd}
\end{subfigure}
\center

\caption{The full process of computing the image of $S=\{10\}$. }
\label{imgprocess}
\end{figure}

Indeed, if we look at the resulting BDD in Figure \ref{resultingBdd}, it symbolically represents the set $\{01,11\}$, which exactly corresponds to our explicit findings earlier in Figure \ref{imgpreimg} ($\{1,3\}$).


\end{document}
%%% Local Variables:
%%% mode: latex
%%% TeX-master: "../master/master"
%%% End:
