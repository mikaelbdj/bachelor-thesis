\documentclass[../master/master.tex]{subfiles}
\begin{document}

This section dsicusses the notions used in the two algorithms we will be focusing on. For the following definitons, let $G=(V,E)$ be a directed graph.

In this thesis, our primary data structure will be a Binary Decision Diagram (BDD). Binary Decision Diagrams represent boolean functions as directed graphs. Each non-terminal vertex $v$ has edges to two children. The two edges $lo(v)$ and $hi(v)$ correspond to the cases where the variable of $v$ is assigned to 0 and 1 respectively. A path from the root to a vertex gives us the value returned by the function given a certain assignment to the variables. \todo{add ref to first paper}

We will be dealing with symbolic algorithms for strongly connected component analysis on these BDDs. A strongly connected component (SCC) is a maximal set of nodes $C\in V$ such that $\forall v,w\in C$, there is a path from $v$ to $w$. Symbolic algorithms manipulate sets of nodes by using symbolic operators. For a directed graph G and $S\subseteq V$, we can derive the following operators: 
\begin{align*}
\img S &= \set{v' \in V}{\exists v\in S:(v, v')\in E}{|}\\
\pre S &= \set{v' \in V}{\exists v\in S:(v', v)\in E}{|}
\end{align*}

The image and preimage operations will be refered to as ``steps'', and the complexity of the algorithms will be measured in terms of these steps. \todo{more symbolic operators in Bloem}
From these operators, we can derive the notion of forward and backward sets, which will be key elements in the algorithms.

A forward set, \FW{S}, is the set of nodes that can be reached from $S$. Similarily, a backward set, \BW{S}, is a set if all the nodes that reach $S$. These can be expressed formally as the least fixpoint of respectively $\lambda S.\{v\}\cup \text{img}(S)$ and $\lambda S.\{v\}\cup \text{preimg}(S)$. Using mu-calculus \cite{clarke_peled_grumberg_1999} this can be expressed as 
\begin{align}\label{fw}
\FW{S} &= \mu S.\{v\}\cup \text{img}(S)\\
\label{bw}\BW{S} &= \mu S.\{v\}\cup \text{preimg}(S)
\end{align}

The authors \cite{linear} introduce a notion of a skeleton and a spine-set. A skeleton of a forward set of the vertex $v$ \FW{v} is \pair{S}{v}, iff the distance from $v$ to $u$ is maximum and $S$ is the set of nodes on the shortest path between the two vertices.

A spine-set in turn is a symbolic way of ordering sets of nodes. A spine-set is a subset \pair{S}{v} of G, for which it contains a chordless path for the set $S$ ending in $v$.

If \FW{v} is the forward-set of $v \in V$, and if the skeleton of this forward set is \pair{S}{u}, then it is also a spine-set in G. We will be making use of this in the linear-time algorithm when computing the spine-set, which is part of the input. 

\end{document}
%%% Local Variables:
%%% mode: latex
%%% TeX-master: "../master/master"
%%% End:
