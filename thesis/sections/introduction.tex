\documentclass[../master/master.tex]{subfiles}

\begin{document}
Graphs are one of the cornerstones in computer science when it comes to representing the models we work with. 

They are used in data organization,  program compilation (registry allocation etc.), networks of communication and in particular model checking. One of the main obstacles in model checking is so called \textit{state explosion}, where the state space grows exponentially in the size of the model. Thus, model checking often has to deal with extremely large graphs \cite{pelanek_2004}. As our model size increases, running an algorithm on the resulting graph quickly becomes infeasible as we cannot store the graph or the computations in memory. A solution to this problem is to find a more concise representation of graphs - one that does not have an exponential increase in storage usage even if the graph size increases exponentially. 

\cite{tarjan_1971}
\end{document}