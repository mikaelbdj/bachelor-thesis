\documentclass[../master.tex]{subfiles}
\newcommand{\FW}[2][G]{\ensuremath{FW_{#1}(#2)}}
\newcommand{\BW}[2][G]{\ensuremath{BW_{#1}(#2)}}
\newcommand{\pair}[2]{\ensuremath{\langle #1, #2\rangle}}

\begin{document}
\subsection{Linear time algorithm}
The second algortihm we will be comparing is the linear time algorithm \cite{linear}. First, we introduce some terms that this algorithm builds upon. We will then look at the algorithm and its complexity analysis. For the following definitons, let $G=(V,E)$ be a digraph.

The key elements of the algorithm are forward and backward sets. Let $U\subseteq V$. A forward set, \FW{U}, is the set of nodes that can be reached from U. Similarily, a backward set, \BW{U}, is a set if all the nodes that reach U.

A skeleton of a forward set of the vertex $v$ \FW{v} is $\langle S, v \rangle$, iff the distance from $v$ to $u$ is maximum and $S$ is the set of nodes on the shortest path between the two vertices

The authors of this algorithm introduce the notion of a spine-set, whih is a symbolic way of ordering sets of nodes. A spine-set is a subset $\langle S, v\rangle$ of a graph $G=(V,E)$, such that $S\subseteq V$, for which G contains a cordless path for the set $S$ ending in $v$.

Having defined some of the notions used in the algorithm, we can proceed to looking at the algorithm described by Gentilini et al. This algorithm takes as input a graph G and spine-set $\langle S, N\rangle$ and outputs vertex sets of SCC subgraphs.
%In each iteration of the algorithm, the SCC of a node is found by computing \FW{v}, and determining which of the vertices in this set have a path back to $v$.
The first thibng the algorithm does, is to choose for which vertex the next SCC will be computed, unless $V=\empty$, in which case the algorithm terminates. If $S\neq\empty$, and $N={v_p}$, $v_p$ is chosen. Otherwise, an arbitrary v is assigned to N. Next, \FW{N} and a skeleton \pair{S'}{u'} on this forward-set is computed.

\todo[inline]{add linear pseudocode in appendix?}
\todo[inline]{is preimage the same as pre here?]}


\end{document}
%%% Local Variables:
%%% mode: latex
%%% TeX-master: "../master/master"
%%% End:
