\documentclass[../master/master.tex]{subfiles}

\begin{document}
\begin{abstract}
  We investigate the performance of two symbolic algorithms for strongly connected component decomposition: the Lockstep algorithm of Bloem et al. and the Linear algorithm of Gentilini et al. Given a graph G = (V,E), their respective worst-case time complexities are O(|V|) and O(|V|log|V|). These bounds are described and tightened in a theoretical analysis. Bloem et al. proposed to \emph{trim} singleton SCCs to improve theperformance of their Lockstep algorithm in practice. We have applied the same technique to the Linear algorithm and also propose our own technique of \emph{edge restriction} inspired by Gentilini et al.
  The performance of the two algorithms, with and without the proposed optimisations, are compared both with respect to the number of symbolic steps performed and the practical running time. These experiments suggest that the linear algorithm performs best in practice, and that \textit{trimming} always improves performance.Edge restriction should be used with more caution: while it improves the practical runtime in the average case, in some cases it exhibits a slowdown. Identifying the types of graphs where edge restriction is of benefit, in practice and/or in theory, is left as possible future work.
\end{abstract}
\end{document}
%%% Local Variables:
%%% mode: latex
%%% TeX-master: "../master/master"
%%% End:
