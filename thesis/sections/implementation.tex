\documentclass[../master/master.tex]{subfiles}
\begin{document}

The description of the implementation of the two algorithms, as well as the framework for them can be seen below. When conveting the psuedocode into Java code, we had to make some small changes. The algorithms decscribed in the two papers \cite{lockstep}\cite{linear} use recursion, however when testing during the process, we experineced that this meant stack overflow on large datasets. We have therefore chosen to use the iterative approach.

\subsection{The JavaBDD library}
An early decision was made to use the JavaBDD library when implementing the algorithms. \cite{whaley}

\subsection{The bddgraph framework}
Before going into the implementation of the algorithms themselves, it is necessary to establish a code foundation that allows us to do the following:
\begin{itemize}
\item Generate a BDD representing a graph, given some conventional graph representation (adjacencylists or edge list).
\item Given a BDD representing a set of nodes, easily compute the image and pre-image of a BDD representing a graph.
\item Pick a BDD representing a node from a BDD representing a graph.
\end{itemize}

With these requirements in mind the code base that can be seen on Figure \ref{main_uml} has been developed. The \texttt{BddGraph} class acts as a wrapper for two \texttt{javabdd.BDD} (from now on BDD) objects: representing nodes and edges respectively. In short, the purpose of the class is to provide an 'api' for programming the symbollic graph algorithms - by providing the nice functions described above. The \texttt{BddGraph} class makes use of some internal classes that are useful abstractions for traditional edges (\texttt{Edge.java}) and for binary numbers (\texttt{Binary.java}).\\

The symbollic algorithms for computing SCC all implement the interface \texttt{GraphSCCAlgorithm}. The sole method in this interface is a method run, that takes a \texttt{BddGraph} object and returns a Set of BDDs. Each BDD represents a set of nodes, so this returned value is exactly the set of SCCs. This interface serves as an abstraction for the algorithm caller - they should not care about the implementation of the run method; just that it returns the SCCs of the graph they provide. 

\begin{figure}[h!]
\begin{center}
\begin{tikzpicture}
\umlemptypackage[x=3]{javabdd}
\begin{umlpackage}{bddgraph}
\umlclass[x=4, below = 1cm of javabdd.south east]{BddGraph}{
  -- bddFactory : BDDFactory \\
  -- integerBinaryMap : Map<Integer,Binary>\\
  -- nodes : BDD\\
  -- edges : BDD\\
  -- v : int
}{
\hspace{1em}«Create»\\ 
+ BddGraph(edges : List<Edge>, verticesCount : int) : BddGraph \\
+ BddGraph(adjacencyLists : List<Queue<Integer>>) : BddGraph 
\vspace{0.3em} \\ 
+ getEdges() : BDD \\
+ getNodes() : BDD \\
+ img(nodeSet : BDD) : BDD \\
+ preImg(nodeSet : BDD) : BDD \\
+ pick(nodeSet : BDD) : BDD
}
\umlclass[below left= 0.5cm of BddGraph.south]{Edge}{
-- from : int\\
-- to : int
}
{
\hspace{1em}«Create»\\ 
+ Edge(from : int, to : int) : Edge \vspace{0.3em} \\ 
+ getFrom() : int\\
+ getTo() : int
}
\umlclass[below right= 0.5cm of BddGraph.south]{Binary}{
-- bools : List<Boolean>
}
{
\hspace{1em}«Create»\\ 
+ Binary(bools : List<Boolean>) : Binary \vspace{0.3em} \\ 
+ getIth(i : int) : Boolean
}
\end{umlpackage}
\begin{umlpackage}{algorithms}
\umlinterface[below = 1.5cm of Edge.south]{GraphSCCAlgorithm}{}{
run(graph : BddGraph) : Set<BDD> 
}

\umlclass[below left= 0.8cm of GraphSCCAlgorithm.south]{Lockstep}{}{}
\umlclass[below right= 0.8cm of GraphSCCAlgorithm.south]{Linear}{}{}
\umlclass[below = 2.4cm of GraphSCCAlgorithm.south]{LockstepWithTrimming}{}{}



\end{umlpackage}

\umluniassoc[geometry=--]{BddGraph}{Edge}
\umluniassoc[geometry=--]{BddGraph}{Binary}
\umlHVHaggreg[arm1=-5cm, thick]{GraphSCCAlgorithm}{BddGraph}
\umlimpl[geometry=--]{Lockstep}{GraphSCCAlgorithm}
\umlimpl[geometry=--]{Linear}{GraphSCCAlgorithm}
\umlimpl[geometry=--]{LockstepWithTrimming}{GraphSCCAlgorithm}
\umlimport[geometry=|-, thick, name=import2]{bddgraph}{javabdd}
\umlimport[geometry=-|, thick, name=import1]{algorithms}{bddgraph}
\umlnote [above right= 1cm of import1-1.base east, geometry=|-] {import1-1}{imports package}
\umlnote [above right= 1cm of import2-1.base east, geometry=|-] {import2-1}{imports package}

\end{tikzpicture}
\end{center}
\caption{UML-diagram describing the implementation of the bddgraph package and it's relation to the javabdd package and the algorithms package}
\label{main_uml}
\end{figure}

\subsection{Lockstep}
With the framework described above in place, implementation of the lockstep algorithm was fairly simple and without much hassle. The code itself can be seen in \todo{(refer to appendix or github repo?)}

Slight optimizations have been made from the version described in the pseudocode in \cite{lockstep}. In their pseudocode (page 45) lines 29-34, they only extend one frontier while the other one is empty. Instead of calling img/preimg on empty sets, we simply just check which set has converged and only extend the other frontier. We have also chosen to restrict edges before doing image computations in Lockstep. The algorithm already contains a restriction on vertices, but this is done after image computations by taking the intersection with P. We were inspired by the edge restriction in linear. This change does not change the number of symbolic operations, however it slightly speeds up the time it takes to compute the image.
Other than this small change, everything is done according to the pseudocode provided. 

\subsubsection*{Trimming}
When describing the lockstep algorithm\cite{lockstep}, the authors mention that the algorithm can be slightly optimized by trimming the set $P$. Trimming refers to removing singleton SCCs, which is done in every call to lockstep. These are of course added to the set of SCCs before being removed.

We have also implemented this variant as we believe it would be interesting to compare it to the original while testing. 

\subsection{Linear}
Just as with implementing the Lockstep algorithm, it was quite straightforward to implement the linear-time algorithm given the framework described and the pseudocode. Most of the code was implemented from the pseudocode of the Linear-time algorithm and the skelForward algorithm.

However, we also needed to implement a set difference method, which was not part of out BDD package. It was then used both in the algorithm, as well as the skelForward helper method.

While implementing the algorithm, we noticed that it is important to include restriction of edges, even though we do not explicitly use them for any computations. In each iteration, we make a new BDD graph from the subsets of vertices and restricted edges, even though it isn't very efficient. The graph is used for the image computations, where it is crucial that the edges are restricted, otherwise the algorithm would loop on some of the same subsets and find already found SCCs, which should never happen.


\end{document}
%%% Local Variables:
%%% mode: latex
%%% TeX-master: "../master/master"
%%% End:
