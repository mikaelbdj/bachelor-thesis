\documentclass[../master/master.tex]{subfiles}
\begin{document}

The description of the implementation of the two algorithms, as well as the framework for them can be seen below. The code described can be found in our \href{https://github.com/mikaelbdj/bachelor-thesis}{GitHub repo}. When converting the psuedocode into Java code, we had to make some small changes. The algorithms decscribed in the two papers \cite{lockstep}\cite{linear} use recursion, however when testing during the process, we experineced that this meant stack overflow on large datasets. We have therefore chosen to use the iterative approach, which can be seen below.

What this means for the algorithms is that first we push the algorithm's arguments to a stack. We then start a while loop while there is something on the stack. We pop these arguments and run the algorithm on them, appending the SCC we have found to our set of SCCs. We then split the graph into two subgraphs, using the appropriate splitting method. We push these two subgraphs to the stack and we run the loop again. At last, we return the SCCs we have found. This effectively simulates the recursive approach, but cuts away all unnecessary overhead incurred by a recursive implementation.

\begin{algorithm}[H]
  \caption{sccAlgorithm($arguments$)}
  \begin{algorithmic}[1]
    \Let{stack}{$arguments$}
    \Statex
    \While{stack $\neq\emptyset$}
    \Let{$arguments$}{pop(stack)}
    \State \(\triangleright\) Run an iteration of the algorithm on the arguments and find SCC
    \Let{SCCs}{SCCs$\cup$SCC}
    \State \(\triangleright\) Split graph according to algorithm into graph subsets $subset1$ and $subset2$
    \State push(stack, $subset1$)
    \State push(stack, $subset2$)
    \EndWhile
    \Statex
    \State \Return SCCs
  \end{algorithmic}
\end{algorithm}

\subsection{The JavaBDD library}
An early decision was made to use the JavaBDD library when implementing the algorithms. \cite{whaley}

\subsection{The bddgraph framework}
Before going into the implementation of the algorithms themselves, it is necessary to establish a code foundation that allows us to do the following:
\begin{itemize}
\item Generate a BDD representing a graph, given some conventional graph representation (adjacencylists or edge list).
\item Given a BDD representing a set of nodes, easily compute the image and pre-image of a BDD representing a graph.
\item Pick a BDD representing a node from a BDD representing a graph.
\end{itemize}

With these requirements in mind the code base that can be seen on Figure \ref{main_uml} has been developed. The \texttt{BddGraph} class acts as a wrapper for two \texttt{javabdd.BDD} (from now on BDD) objects: representing nodes and edges respectively. In short, the purpose of the class is to provide an 'api' for programming the symbollic graph algorithms - by providing the nice functions described above. The \texttt{BddGraph} class makes use of some internal classes that are useful abstractions for traditional edges (\texttt{Edge.java}) and for binary numbers (\texttt{Binary.java}).\\

The symbollic algorithms for computing SCC all implement the interface \texttt{GraphSCCAlgorithm}. The sole method in this interface is a method run, that takes a \texttt{BddGraph} object and returns a Set of BDDs. Each BDD represents a set of nodes, so this returned value is exactly the set of SCCs. This interface serves as an abstraction for the algorithm caller - they should not care about the implementation of the run method; just that it returns the SCCs of the graph they provide.

\begin{figure}[H]
\begin{center}
\begin{tikzpicture}
\umlemptypackage[x=3]{javabdd}
\begin{umlpackage}{bddgraph}
\umlclass[x=4, below = 1cm of javabdd.south east]{BddGraph}{
  -- bddFactory : BDDFactory \\
  -- integerBinaryMap : Map<Integer,Binary>\\
  -- nodes : BDD\\
  -- edges : BDD\\
  -- v : int
}{
\hspace{1em}«Create»\\ 
+ BddGraph(edges : List<Edge>, verticesCount : int) : BddGraph \\
+ BddGraph(adjacencyLists : List<Queue<Integer>>) : BddGraph 
\vspace{0.3em} \\ 
+ getEdges() : BDD \\
+ getNodes() : BDD \\
+ img(nodeSet : BDD) : BDD \\
+ preImg(nodeSet : BDD) : BDD \\
+ pick(nodeSet : BDD) : BDD
}
\umlclass[below left= 0.5cm of BddGraph.south]{Edge}{
-- from : int\\
-- to : int
}
{
\hspace{1em}«Create»\\ 
+ Edge(from : int, to : int) : Edge \vspace{0.3em} \\ 
+ getFrom() : int\\
+ getTo() : int
}
\umlclass[below right= 0.5cm of BddGraph.south]{Binary}{
-- bools : List<Boolean>
}
{
\hspace{1em}«Create»\\ 
+ Binary(bools : List<Boolean>) : Binary \vspace{0.3em} \\ 
+ getIth(i : int) : Boolean
}
\end{umlpackage}
\begin{umlpackage}{algorithms}
\umlinterface[below = 1.5cm of Edge.south]{GraphSCCAlgorithm}{}{
run(graph : BddGraph) : Set<BDD> 
}

\umlclass[below left= 0.8cm of GraphSCCAlgorithm.south]{Lockstep}{}{}
\umlclass[right= 0.8cm of Lockstep]{Linear}{}{}
\umlclass[right = 0.8 cm of Linear]{LinearTrim}{}{}
\umlclass[below = 0.8 cm of Lockstep]{LockstepER}{}{}
\umlclass[right = 0.8 cm of LockstepER]{LockstepERTrim}{}{}



\end{umlpackage}

\umluniassoc[geometry=--]{BddGraph}{Edge}
\umluniassoc[geometry=--]{BddGraph}{Binary}
\umlHVHaggreg[arm1=-5cm, thick]{GraphSCCAlgorithm}{BddGraph}
\umlimpl[geometry=--]{Lockstep}{GraphSCCAlgorithm}
\umlimpl[geometry=--]{Linear}{GraphSCCAlgorithm}
\umlimpl[geometry=--]{LinearTrim}{GraphSCCAlgorithm}
\umlimpl[geometry=-|, name=locksteparrow]{LockstepER}{GraphSCCAlgorithm}
\umlimpl[geometry=-|]{LockstepERTrim}{GraphSCCAlgorithm}
\umlimport[geometry=|-, thick, name=import2]{bddgraph}{javabdd}
\umlHVHimport[arm1=10cm, thick, name=import1]{algorithms}{bddgraph}
\umlnote [above left= 1cm of import1-2.base east, geometry=--] {import1-2}{imports package}
\umlnote [above right= 1cm of import2-1.base east, geometry=|-] {import2-1}{imports package}

\end{tikzpicture}
\end{center}
\caption{UML-diagram describing the implementation of the bddgraph package and it's relation to the javabdd package and the algorithms package}
\label{main_uml}
\end{figure}

\subsection{Lockstep}
With the framework described above in place, implementation of the lockstep algorithm was fairly simple and without much hassle. As we progressed, we engineered some different optimizations from the original Lockstep implementation. These will all be experimented upon. Here we will shortly describe them.

\subsubsection{Vanilla Lockstep}
Slight optimizations have been made from the version described in the pseudocode in \cite{lockstep}. In their pseudocode (page 45) lines 29-34, they only extend one frontier while the other one is empty. Instead of calling img/preimg on empty sets, we simply just check which set has converged and only extend the other frontier.
Other than this small change (shown below), everything is implemented according to the pseudocode shown in the earlier section. In Figure \ref{main_uml} this implementation is in the class \texttt{Lockstep}.

\begin{algorithm}[H]
  \caption{Second While Loop}
  \begin{algorithmic}
    \If{Ffront=$\emptyset$}
    \Let{$Converged$}{$FW$}
    \While{$Bfront\cap FW \neq \emptyset$}
      \Let{Bfront}{$\text{preimg(Bfront)}\cap P \setminus BW$}
      \Let{$BW$}{$BW \cup \text{Bfront}$}
      \EndWhile
      \Else
      \Let{$Converged$}{$BW$}
    \While{$Ffrontnt\cap BW \neq \emptyset$}
      \Let{Ffront}{$\text{img(Ffront)}\cap P \setminus FW$}
      \Let{$FW$}{$FW \cup \text{Ffront}$}
      \EndWhile
    \EndIf
  \end{algorithmic}
\end{algorithm}

\subsubsection{Edge restriction}
Getting inspired by the linear algorithm, we have implemented a slightly modified version of lockstep in which the edges are restricted before doing image computations. The algorithm already contains a restriction on vertices, but this is done after image computations by taking the intersection with P. This change does not change the number of symbolic steps, but we will show that it can have a significant impact on the size of the steps and thus the final practical runtime of the algorithm. This will be further expanded upon when we experiment. This version of Lockstep has its own class in Figure \ref{main_uml}: \texttt{LockstepER}.

\subsubsection{Trimming}
When describing the lockstep algorithm\cite{lockstep}, the authors mention that the algorithm can be slightly optimized by trimming the set $P$. Trimming refers to removing singleton SCCs, which is done in every call to lockstep. These are of course added to the set of SCCs before being removed.

Trimming has been implemented on top of edge restriction in the class 
\texttt{LockstepERTrim}

\subsection{Linear}
Just as with implementing the Lockstep algorithm, it was quite straightforward to implement the linear-time algorithm given the framework described and the pseudocode. Most of the code was implemented from the pseudocode of the Linear-time algorithm and the skelForward algorithm.

While implementing the algorithm, we noticed that it is important to include restriction of edges, even though we do not explicitly use them for any computations. In each iteration, we make a new BDD graph from the subsets of vertices and restricted edges. The graph is used for the image computations, where it is crucial that the edges are restricted, otherwise the algorithm would loop on some of the same subsets and find already found SCCs, which should never happen.
The linear algorithm has been implemented in the class \texttt{Linear}. 

\subsubsection{Linear with trimming}
As with lockstep, we have added a version of the algorithm that implements trimming. This is implemented on top of the regular linear implementation in the class \texttt{LinearTrim}.




\end{document}
%%% Local Variables:
%%% mode: latex
%%% TeX-master: "../master/master"
%%% End:
