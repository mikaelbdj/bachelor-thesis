\documentclass{article}

% Imports
\usepackage[utf8]{inputenc}
\usepackage[margin=1in]{geometry}
\usepackage{lastpage}
\usepackage{fancyhdr}
\usepackage{graphicx}

% Header
\setlength{\headheight}{48.14pt} 
\fancyhf{}
\fancyhead[OL]{
\includegraphics[scale=0.3]{AUlogo}
\vspace{-0.5cm}
}
\fancyhead[OR]{
\begin{tabular}{l}
\textbf{\sc Bachelor Contract}       \\
Date: \today              \\
Page \thepage/\pageref{LastPage}\\
~~
\end{tabular}}

\newcommand{\timeest}[1]{$\mathbf{#1}$}% used to write time estimations without excessive math-mode

\begin{document}
\pagestyle{fancy}

% Meta-information about group/advisors/etc.
\bgroup\def\arraystretch{1.5}
\begin{table}[h]
\begin{tabular}{ll}
\textbf{Advisor}     & Andreas Pavlogiannis \& Jaco van de Pol               \\
\textbf{Students}    & Magdalena Kalin-Czerska \& Jonathan Eilath \& Mikael Bisgaard Dahlsen-Jensen   \\
\textbf{Languages}   & English \\
\textbf{Text tools}  & \LaTeX       \\
\textbf{Other tools} & VSCode / IntelliJ  / Git       
\end{tabular}
\end{table}
\egroup\vspace{-0.cm}

\subsection*{Project Description (at least 10-20 lines)}

This project will investigate implementations of two different symbolic algorithms. These algorithms for Strongly-connected-components (SCC) decomposition have different theoretical upper-bounds on the amount of symbolic operations. Whether or not this will carry over to practical differences will be explored. Henceforth, a thorough theoretical analysis on these results will be conducted. Depending on the previous results, one might attempt to combine the two algorithms if it seems reasonable. Alternatively, other similar symbollic algorithms will be investigated, mainly focusing on solving other problems than SCC. \\[2mm]





\subsection*{Provisional Table of Contents}
\begin{itemize}
    \item Abstract (10-20 lines)
    \item Section 1: Introduction (1-2 pages)
    \item Section 2: Description of Emerson-Lei and Lockstep algorithm (4 pages)
    \item Section 3: Implementing the algorithms (4 pages)
    \item Section 4: Inputs and experimenting (2 pages)
    \item Section 5: Comparison of results (2 pages)
    \item Section 6: Theoretical analysis of results (4-8 pages)
    \item Section 7: Investigation of other graph problems using symbolic algorithms (2-4 pages)
    \item Section 8: Conclusion (1-2 pages)
    \item Acknowledgements (3-5 lines)
    \item References ($\frac{1}{2}$-1 page)
    \item Appendix with programming code, tables, full proofs, etc. (5-20 pages)
\end{itemize}

\subsection*{Provisional Time Plan}

\paragraph{First week of February (15 hours)}~\\\noindent
Planning of activities, including the production of the Bachelor's contract.

\paragraph{Rest of February and first week of March (\timeest{4\times 15} hours)}~\\\noindent
Read literature and make draft of Section 2 in Bachelor's report.

\paragraph{Rest of March (\timeest{2\times 15+1\times 30} hours)}~\\\noindent
Implement the two algorithms and make draft of Section 3 in Bachelor's report.

\paragraph{First two weeks of April (\timeest{2\times 30} hours)}~\\\noindent
Find inputs, experiment, compare the results and make draft of Section 4 and 5 in Bachelor's report.

\paragraph{Rest of April and first two weeks of May (\timeest{4\times 30} hours)}~\\\noindent
Conduct theoretical analysis and investigate other graph problems. Make draft of Section 6 and 7 in Bachelor's report.

\paragraph{Last two weeks of May and first half of June (\timeest{4\times 30} hours)}~\\\noindent
Write the missing parts, put drafts together, make things consistent, proof reading.

\end{document}
